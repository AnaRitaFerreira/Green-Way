\begin{vdmpp}[breaklines=true]
class Tests is subclass of MyTestCase

operations

(*@
\label{Run:5}
@*)
 public static Run : () ==> ()
  Run()  ==
  (
  
  -- Valid test cases  
  Test1(); --testar adicao e remocao de protocolos /*R1*/
  Test2(); --testar adicao e remocao de clientes /*R2*/
  Test3(); --testar passagem em sensor de highway (entrada e saida) /*R3*/
  Test4(); --testar passagem em sensor de pagamento simples /*R3*/
  Test5(); --testar passagem em sensor de pagamento complexo (dependente de classe do veiculo) /*R3*/
  Test6(); --testar passagem em sensores de parque de estacionamento (entrada e saida) /*R3*/
  Test7(); --testar emissao e pagamento de faturas /*R4+R5*/
  
  -- Error test cases (remover o coment�rio 1 a 1 para testar que falham)   
  -- Test8(); --testar remocao de protocolos que nao existem ou adicao multipla /*R1*/
  -- Test9(); --testar a remocao de clientes que nao existam ou adicao multipla /*R2*/
  -- Test10(); --testar passar num par de sensores (entrada-saida) que nao existe na tabela de pre�os /*R3*/
   -- Test11(); --testar passar em sensores de highway violando restricoes espa�o-temporais (velocidade 3) /*R3*/
    -- Test12(); --testar passar em num sensor com tempo anterior /*R3*/
    -- Test13(); --testar passar em num no que nao existe em nenhum servi�o /*R3*/
    -- Test14(); --testar passar em parque saida antes de entrada /*R3*/
    -- Test15(); --testar passar em parque entrar duas vezes (entradas diferentes) /*R3*/
    -- Test16();--testar entrar num parque e passar na autoestrada a meio /*R3*/
    -- Test17(); --testar passar ao mesmo tempo em 2 nos /*R3*/
    -- Test18(); --testar pagar fatura que nao foi emitida /*R5*/
  );
  

  /* VALID TEST CASES */
  
  --testar adicao e remocao de protocolos
(*@
\label{Test1:36}
@*)
  public static Test1 : () ==> ()
  Test1() ==
  (
   dcl greenWay : Green_Way;
   dcl brisa: Service_Provider;
   dcl ascendi: Service_Provider;
   
   greenWay := new Green_Way();
   
   brisa := new Service_Provider("Brisa");
   ascendi := new Service_Provider("Ascendi");
   
   greenWay.addServiceProvider(brisa);
   greenWay.addServiceProvider(ascendi);
   greenWay.removeServiceProvider(brisa);
   assertEqual(greenWay.sproviders, {ascendi});
   
  );
    
  --testar adicao e remocao de clientes
(*@
\label{Test2:56}
@*)
  public static Test2 : () ==> ()
  Test2() ==
  (
   dcl greenWay : Green_Way;
   dcl c1: Client;
   dcl c2: Client;
   dcl c3: Client;
   
   greenWay := new Green_Way();
   
   c1 := new Client(<C1>, "francisco", 281923918);
   c2 := new Client(<C4>, "joao", 1236334198);
   c3 := new Client(<C3>, "hugo", 1236334198);
   
   greenWay.addClient(c1);
   greenWay.addClient(c3); 
   greenWay.removeClient(c1);
   greenWay.addClient(c1);
   greenWay.addClient(c2);
   greenWay.removeClient(c3);
   
   assertEqual(greenWay.clients, {c1,c2});
    
   
  );
  
  --testar passagem em sensor de highway (entrada e saida)
(*@
\label{Test3:83}
@*)
  public static Test3 : () ==> ()
  Test3() ==
  (
   dcl greenWay : Green_Way;
   dcl c1: Client;
   dcl brisa: Service_Provider;
   dcl AE_entrada: Highway;
   dcl AE_saida: Highway;
   dcl no: Spot;
    --dcl sensor: SinglePoint;
   
   greenWay := new Green_Way();
   AE_entrada := new Highway(5,5, <ENTRANCE>);
    AE_saida := new Highway(6,6, <EXIT>);
    no := new Spot(5,5);
    
   greenWay.highway_prices := 
  { 
   mk_Green_Way`OriginDestiny(AE_entrada, AE_saida) |-> {<C1> |-> 4.2, <C2> |-> 5.6, <C3> |-> 5.6, <C4> |-> 5.6}
  };
   
   c1 := new Client("francisco", 281923918);
   greenWay.addClient(c1);
   brisa := new ServiceProvider("Brisa");
    --sensor := new SinglePoint(no, 2.3);
   brisa.addService(sensor);  
   
   greenWay.addServiceProvider(brisa);
   greenWay.passa(c1, AE_entrada, new Time(10));
   greenWay.passa(c1, no, new Time(20));
   greenWay.passa(c1, AE_saida, new Time(200));
   
   --1o registo: entrada / 2o registo: saida
   assertEqual(0, greenWay.passages(c1)(1).cost);
   assertEqual(2.3, greenWay.passages(c1)(2).cost);
   assertEqual(4.2, greenWay.passages(c1)(3).cost);
   
  );
  
  --testar passagem em sensor de pagamento simples
(*@
\label{Test4:123}
@*)
  public static Test4 : () ==> ()
  Test4() ==
  (
   dcl greenWay : Green_Way;
   dcl c1: Client;
   dcl brisa: Service_Provider;
   dcl no: Spot;
   dcl sensor: SinglePoint;
   
   greenWay := new greenWay();
   no := new Node(5,5);
   
   c1 := new Client(<C1>, "francisco", 281923918);
   greenWay.addClient(c1);
   brisa := new ServiceProvider("Brisa");
   sensor := new SinglePoint(no, 2.3);
   brisa.addService(sensor);
   greenWay.addServiceProvider(brisa);
   
   greenWay.passa(c1, no, new Time(10));
   
   --1o registo: entrada / 2o registo: saida
   assertEqual(2.3, greenWay.records(c1)(1).cost);
   
  );
  
  --testar passagem em sensor de pagamento complexo (dependente de classe do veiculo)
(*@
\label{Test5:150}
@*)
  public static Test5 : () ==> ()
  Test5() ==
  (
   dcl greenWay : greenWay;
   dcl c1: Client;
   dcl c2: Client;
   dcl c3: Client;
   dcl c4: Client;
   dcl brisa: ServiceProvider;
   dcl no: Node;
   dcl sensor: ComplexSinglePoint;
   
   
   greenWay := new greenWay();
   no := new Node(5,5);
   
   c1 := new Client(<C1>, "francisco", 281923918);
   c2 := new Client(<C2>, "hugo", 123456789);
   c3 := new Client(<C3>, "carla", 738920139);
   c4 := new Client(<C4>, "cristiano", 788832193);
   greenWay.addClient(c1);
   greenWay.addClient(c2);
   greenWay.addClient(c3);
   greenWay.addClient(c4);
   brisa := new ServiceProvider("Brisa");
   sensor := new ComplexSinglePoint(no, {<C1> |-> 1.0, <C2> |-> 2.2, <C3> |-> 3.1, <C4> |-> 4.9});
   brisa.addService(sensor);
   greenWay.addServiceProvider(brisa);
   
   greenWay.passa(c1, no, new Time(10));
   greenWay.passa(c2, no, new Time(11));
   greenWay.passa(c3, no, new Time(12));
   greenWay.passa(c4, no, new Time(13));
   
   assertEqual(1.0, greenWay.records(c1)(1).cost);
   assertEqual(2.2, greenWay.records(c2)(1).cost);
   assertEqual(3.1, greenWay.records(c3)(1).cost);
   assertEqual(4.9, greenWay.records(c4)(1).cost);
   
  );
  
  
  --testar passagem em sensores de parque de estacionamento (entrada e saida)
(*@
\label{Test6:193}
@*)
  public static Test6 : () ==> ()
  Test6() ==
  (
   dcl greenWay : greenWay;
   dcl brisa: ServiceProvider;
   dcl c1: Client;
   dcl entrada_parque : Node;
   dcl saida_parque : Node;
   dcl parque1 : Service;
   
   greenWay := new greenWay();
   c1 := new Client(<C1>, "francisco", 281923918);
   greenWay.addClient(c1);
   brisa := new ServiceProvider("Brisa");
   entrada_parque := new Node(1,1);
   saida_parque := new Node(2,2);
   
   parque1 := new ParkingLot({entrada_parque}, {saida_parque}, 1.5);
   brisa.addService(parque1);
   
   --TODO fazer o check ao adicionar serviceProvider que nenhum dos nos adicionados dentro esta a ser usado noutro serviceProdiver
   greenWay.addServiceProvider(brisa);
   greenWay.passa(c1, entrada_parque, new Time(10));
   greenWay.passa(c1, saida_parque, new Time(15));
    greenWay.passa(c1, entrada_parque, new Time(18));
    
    assertEqual(0, greenWay.records(c1)(1).cost);
    --unidades de tempo * pre�o por unidade de tempo
    assertEqual(5*1.5, greenWay.records(c1)(2).cost);
   
  );
  
    --testar emissao e pagamento de faturas
(*@
\label{Test7:226}
@*)
  public static Test7 : () ==> ()
  Test7() ==
  (
   dcl greenWay : greenWay;
   dcl brisa: ServiceProvider;
   dcl c1: Client;
   dcl entrada_parque : Node;
   dcl saida_parque : Node;
   dcl parque1 : Service;
   
   greenWay := new greenWay();
   c1 := new Client(<C1>, "francisco", 281923918);
   greenWay.addClient(c1);
   brisa := new ServiceProvider("Brisa");
   entrada_parque := new Node(1,1);
   saida_parque := new Node(2,2);
   
   parque1 := new ParkingLot({entrada_parque}, {saida_parque}, 1.5);
   brisa.addService(parque1);
   
   --TODO fazer o check ao adicionar serviceProvider que nenhum dos nos adicionados dentro esta a ser usado noutro serviceProdiver
   greenWay.addServiceProvider(brisa);
   greenWay.passa(c1, entrada_parque, new Time(10));
   greenWay.passa(c1, saida_parque, new Time(15));
    
    --faturas sao emitidas
    greenWay.incrementMonth();
    
    assertEqual(5*1.5, greenWay.invoices(c1)(1).price);
    assertEqual(false, greenWay.invoices(c1)(1).paid);
    
    --month comeca em 1 e ano em 2014 por defeito
    greenWay.payInvoice(c1, 1, 2014);
   
   assertTrue(greenWay.invoices(c1)(1).paid);
  );
  
   /* ERROR TEST CASES */
  
  --testar a remocao de protocolos que nao existem ou adicao multipla
(*@
\label{Test8:266}
@*)
  public static Test8 : () ==> ()
  Test8() ==
  (
   dcl greenWay : greenWay;
   dcl brisa: ServiceProvider;
   dcl ascendi: ServiceProvider;
   
   greenWay := new greenWay();
   
   brisa := new ServiceProvider("Brisa");
   ascendi := new ServiceProvider("Ascendi");
   
   greenWay.addServiceProvider(brisa);
   -- greenWay.addServiceProvider(brisa); -- descomentar para testar adicao multipla
    
   greenWay.removeServiceProvider(brisa);
   greenWay.removeServiceProvider(ascendi);
  );
  
  --testar a remocao de clientes que nao existam ou adiacao multipla
(*@
\label{Test9:286}
@*)
  public static Test9 : () ==> ()
  Test9() ==
  (
   dcl greenWay : greenWay;
   dcl c1: Client;
   dcl c2: Client;
   
   greenWay := new greenWay();
   
   c1 := new Client(<C1>, "francisco", 281923918);
   c2 := new Client(<C4>, "joao", 1236334198);
   
   greenWay.addClient(c1);
   -- greenWay.addClient(c1); --remover comentario pra testar adicao multipla
   
   greenWay.removeClient(c1);
   greenWay.removeClient(c2);

  );
  
  --testar passar em sensores de highway de que nao existem nessa highway
(*@
\label{Test10:307}
@*)
  public static Test10 : () ==> ()
  Test10() ==
  (
   dcl greenWay : greenWay;
   dcl c1: Client;
   dcl AE_entrada: HighwayNode;
   dcl AE_saida: HighwayNode;
   dcl AE_entrada2: HighwayNode;
   dcl AE_saida2: HighwayNode;
   
   greenWay := new greenWay();
   AE_entrada := new HighwayNode(5,5, <ENTRANCE>);
    AE_saida := new HighwayNode(6,6, <EXIT>);
    AE_entrada2 := new HighwayNode(4,4, <ENTRANCE>);
    AE_saida2 := new HighwayNode(6,8, <EXIT>);
    
   greenWay.highway_prices := 
  { 
   mk_greenWay`OriginDestiny(AE_entrada, AE_saida) |-> {<C1> |-> 4.2, <C2> |-> 5.6, <C3> |-> 5.6, <C4> |-> 5.6},
   mk_greenWay`OriginDestiny(AE_entrada2, AE_saida2) |-> {<C1> |-> 4, <C2> |-> 5, <C3> |-> 8, <C4> |-> 10}
   
  };
   
   c1 := new Client(<C1>, "francisco", 281923918);
   greenWay.addClient(c1);
   
   
   greenWay.passa(c1, AE_entrada, new Time(10));
   greenWay.passa(c1, AE_saida2, new Time(200));

   
  );
  
   --testar passar em sensores de highway violando restricoes espa�o-temporais (velocidade 3)
(*@
\label{Test11:341}
@*)
  public static Test11 : () ==> ()
  Test11() ==
  (
   dcl greenWay : greenWay;
   dcl c1: Client;
   dcl AE_entrada: HighwayNode;
   dcl AE_saida: HighwayNode;
   
   greenWay := new greenWay();
   AE_entrada := new HighwayNode(5,5, <ENTRANCE>);
    AE_saida := new HighwayNode(5,12, <EXIT>); 
    -- distancia de 7, deveria ser no maximo 6 (em 2 unidades de tempo)    
   greenWay.highway_prices := 
  { 
   mk_greenWay`OriginDestiny(AE_entrada, AE_saida) |-> {<C1> |-> 4.2, <C2> |-> 5.6, <C3> |-> 5.6, <C4> |-> 5.6}
  };
   
   c1 := new Client(<C1>, "francisco", 281923918);
   greenWay.addClient(c1);

   greenWay.passa(c1, AE_entrada, new Time(10));
   greenWay.passa(c1, AE_saida, new Time(12));
   /* -- Outros exemplos
   greenWay.passa(c1, AE_entrada, new Time(10));
   greenWay.passa(c1, AE_entrada, new Time(12));
   --
   greenWay.passa(c1, AE_saida, new Time(10));
   greenWay.passa(c1, AE_saida, new Time(12));
   */
  );
  
(*@
\label{Test12:372}
@*)
    public static Test12 : () ==> ()
  Test12() ==
  (
   dcl greenWay : greenWay;
   dcl c1: Client;
   dcl AE_entrada: HighwayNode;
   dcl AE_saida: HighwayNode;
   
   greenWay := new greenWay();
   AE_entrada := new HighwayNode(5,5, <ENTRANCE>);
    AE_saida := new HighwayNode(5,6, <EXIT>); 

   greenWay.highway_prices := 
  { 
   mk_greenWay`OriginDestiny(AE_entrada, AE_saida) |-> {<C1> |-> 4.2, <C2> |-> 5.6, <C3> |-> 5.6, <C4> |-> 5.6}
  };
   
   c1 := new Client(<C1>, "francisco", 281923918);
   greenWay.addClient(c1);

   greenWay.passa(c1, AE_entrada, new Time(10));
   greenWay.passa(c1, AE_saida, new Time(6));

  );
  
  --testar passar em num no que nao existe em nenhum servi�o
(*@
\label{Test13:398}
@*)
  public static Test13 : () ==> ()
  Test13() ==
  (
   dcl greenWay : greenWay;
   dcl c1: Client;
   dcl brisa: ServiceProvider;
   dcl AE_entrada: HighwayNode;
   dcl AE_saida: HighwayNode;
   dcl entrada_parque : Node;
   dcl saida_parque: Node;
   dcl parque1: ParkingLot;
   dcl solo_node: Node;
   
   greenWay := new greenWay();
   AE_entrada := new HighwayNode(5,5, <ENTRANCE>);
    AE_saida := new HighwayNode(5,6, <EXIT>); 
    entrada_parque := new Node(1,1);
   saida_parque := new Node(2,2);
   solo_node:= new Node(5,5);
   
   parque1 := new ParkingLot({entrada_parque}, {saida_parque}, 1.5);
   brisa := new ServiceProvider("Brisa");
   brisa.addService(parque1);   
   greenWay.addServiceProvider(brisa);
   
   greenWay.highway_prices := 
  { 
   mk_greenWay`OriginDestiny(AE_entrada, AE_saida) |-> {<C1> |-> 4.2, <C2> |-> 5.6, <C3> |-> 5.6, <C4> |-> 5.6}
  };
   
   c1 := new Client(<C1>, "francisco", 281923918);
   greenWay.addClient(c1);

   greenWay.passa(c1, AE_entrada, new Time(10));
   greenWay.passa(c1, AE_saida, new Time(15));
   greenWay.passa(c1, entrada_parque, new Time(20));
   greenWay.passa(c1, saida_parque, new Time(25));
    greenWay.passa(c1, solo_node, new Time(30));
   

  );
  --testar passar em parque saida antes de entrada
(*@
\label{Test14:440}
@*)
  public static Test14 : () ==> ()
  Test14() ==
  (
   dcl greenWay : greenWay;
   dcl brisa: ServiceProvider;
   dcl c1: Client;
   dcl e1 : Node;
   dcl e2 : Node;
   dcl s1 : Node;
   dcl s2 : Node;
   
   dcl parque1 : Service;
   
   greenWay := new greenWay();
   c1 := new Client(<C1>, "francisco", 281923918);
   greenWay.addClient(c1);
   brisa := new ServiceProvider("Brisa");
   e1 := new Node(1,1);
   s1 := new Node(1,1);
   e2 := new Node(2,2);
   s2 := new Node(2,2);
   
   parque1 := new ParkingLot({e1,e2}, {s1,s2}, 1.5);
   brisa.addService(parque1);
   
   --TODO fazer o check ao adicionar serviceProvider que nenhum dos nos adicionados dentro esta a ser usado noutro serviceProdiver
   greenWay.addServiceProvider(brisa);
   greenWay.passa(c1, s1, new Time(10));
   greenWay.passa(c1, s2, new Time(15));
    
  );
  
  --testar passar em parque entrar duas vezes (entradas diferentes)
(*@
\label{Test15:473}
@*)
    public static Test15 : () ==> ()
  Test15() ==
  (
   dcl greenWay : greenWay;
   dcl brisa: ServiceProvider;
   dcl c1: Client;
   dcl e1 : Node;
   dcl e2 : Node;
   dcl s1 : Node;
   dcl s2 : Node;
   
   dcl parque1 : Service;
   
   greenWay := new greenWay();
   c1 := new Client(<C1>, "francisco", 281923918);
   greenWay.addClient(c1);
   brisa := new ServiceProvider("Brisa");
   e1 := new Node(1,1);
   s1 := new Node(1,1);
   e2 := new Node(2,2);
   s2 := new Node(2,2);
   
   parque1 := new ParkingLot({e1,e2}, {s1,s2}, 1.5);
   brisa.addService(parque1);
   
   --TODO fazer o check ao adicionar serviceProvider que nenhum dos nos adicionados dentro esta a ser usado noutro serviceProdiver
   greenWay.addServiceProvider(brisa);
   greenWay.passa(c1, e1, new Time(10));
   greenWay.passa(c1, e2, new Time(15));
   
   --2 saidas apos entrada tambem falha
   --greenWay.passa(c1, s1, new Time(20));
   --greenWay.passa(c1, s1, new Time(21));
   
    
  );
  --testar entrar num parque e passar na autoestrada a meio
(*@
\label{Test16:510}
@*)
      public static Test16 : () ==> ()
  Test16() ==
  (
   dcl greenWay : greenWay;
   dcl brisa: ServiceProvider;
   dcl c1: Client;
   dcl e1 : Node;
   dcl e2 : Node;
   dcl s1 : Node;
   dcl s2 : Node;
   dcl AE_entrada: HighwayNode;
   dcl AE_saida: HighwayNode;
   dcl parque1 : Service;
   
   greenWay := new greenWay();
   c1 := new Client(<C1>, "francisco", 281923918);
   greenWay.addClient(c1);
   brisa := new ServiceProvider("Brisa");
   e1 := new Node(1,1);
   s1 := new Node(1,1);
   e2 := new Node(2,2);
   s2 := new Node(2,2);
    AE_entrada := new HighwayNode(5,5, <ENTRANCE>);
    AE_saida := new HighwayNode(5,12, <EXIT>); 
    
       greenWay.highway_prices := 
  { 
   mk_greenWay`OriginDestiny(AE_entrada, AE_saida) |-> {<C1> |-> 4.2, <C2> |-> 5.6, <C3> |-> 5.6, <C4> |-> 5.6}
  };
  
   parque1 := new ParkingLot({e1,e2}, {s1,s2}, 1.5);
   brisa.addService(parque1);
   
   --TODO fazer o check ao adicionar serviceProvider que nenhum dos nos adicionados dentro esta a ser usado noutro serviceProdiver
   greenWay.addServiceProvider(brisa);
   greenWay.passa(c1, e1, new Time(10));
   greenWay.passa(c1, AE_saida, new Time(300));

   
    
  );
  
  --testar entrar e sair de parque no mesmo timestamp
(*@
\label{Test17:553}
@*)
  public static Test17 : () ==> ()
  Test17() ==
  (
   dcl greenWay : greenWay;
   dcl brisa: ServiceProvider;
   dcl c1: Client;
   dcl e1 : Node;
   dcl e2 : Node;
   dcl s1 : Node;
   dcl s2 : Node;
   
   dcl parque1 : Service;
   
   greenWay := new greenWay();
   c1 := new Client(<C1>, "francisco", 281923918);
   greenWay.addClient(c1);
   brisa := new ServiceProvider("Brisa");
   e1 := new Node(1,1);
   s1 := new Node(1,1);
   e2 := new Node(2,2);
   s2 := new Node(2,2);
   
   parque1 := new ParkingLot({e1,e2}, {s1,s2}, 1.5);
   brisa.addService(parque1);
   
   --TODO fazer o check ao adicionar serviceProvider que nenhum dos nos adicionados dentro esta a ser usado noutro serviceProdiver
   greenWay.addServiceProvider(brisa);
   greenWay.passa(c1, e1, new Time(10));
   greenWay.passa(c1, s2, new Time(10));
   
    
  );
  
  --testar pagar fatura que nao foi emitida
(*@
\label{Test18:587}
@*)
  public static Test18 : () ==> ()
  Test18() ==
  (
   dcl greenWay : greenWay;
   dcl c1: Client;
   
   greenWay := new greenWay();
   c1 := new Client(<C1>, "francisco", 281923918);
   greenWay.addClient(c1);
   
    --faturas sao emitidas
    greenWay.incrementMonth();
        
    --month comeca em 1 e ano em 2014 por defeito, logo a fatura de 5-2015 nao foi emitida
    greenWay.payInvoice(c1, 5, 2015);   
  );
end Tests
\end{vdmpp}
