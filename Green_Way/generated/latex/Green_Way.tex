\documentclass[a4paper]{article}
\usepackage{fullpage}
\usepackage[color]{vdmlisting}
\usepackage[hidelinks]{hyperref} 
\usepackage{longtable}

\usepackage[portuguese]{babel}
\usepackage[utf8]{inputenc}
\usepackage{indentfirst}
\usepackage{graphicx}
\usepackage{verbatim}
\usepackage{hyperref}
\begin{document}

\title{\Huge\textbf{Green Way System 
}\linebreak\linebreak\linebreak
\Large\textbf{Relatório Final}\linebreak\linebreak
\linebreak\linebreak
\includegraphics[scale=0.1]{feup-logo.png}\linebreak\linebreak
\linebreak\linebreak\linebreak
\linebreak
\Large{Mestrado Integrado em Engenharia Informática e Computação} \linebreak\linebreak
\Large{Métodos Formais em Engenharia de Software}\linebreak\linebreak
\linebreak
}

\author{Ana Rita Ferreira - 201205014/ei12052@fe.up.pt \\ Jorge Teixeira - 201205117/ei12030@fe.up.pt \\\linebreak\linebreak \\
 \\ Faculdade de Engenharia da Universidade do Porto \\ Rua Roberto Frias, s\/n, 4200-465 Porto, Portugal \linebreak\linebreak\linebreak
\linebreak\linebreak\vspace{1cm}}


\maketitle
\thispagestyle{empty}

%************************************************************************************************

\newpage
\tableofcontents
\newpage

\section{Informal system description and list of requirements}
Requirements should include any relevant constraints (regarding safety, etc.). 
Each requirement should have an identifier. 
You may have optional requirements. 
\section{Visual UML model}
A use case model, describing the system actors and 
use cases, with a short description 
of each major use case. 
One or more class diagram(s), describing the structure of the VDM++ model, with a 
short description of each class, plus any other relevant explanations. 
\section{Formal VDM++ model}
VDM++ classes, properly commented. 
Needed data types (e.g., String, Date, etc.) should be modeled with types, values and 
functions. 
Domain entities should be modeled with classes, ins
tance variables and operations. 
You are expected to make adequate usage of the VDM++ types (sets, sequences, maps, etc.) and create a model at a high level of abstraction. 
The model should contain adequate contracts, i.e., 
invariants, preconditions, and -conditions. Post-conditions need only be defined in cases where they are significantly different from the operation or function body (e.g., t
he post-condition of a sqrt(x) operation, which simply states that x = RESULT * RESULT, should be significantly different than the body); for learning purposes, you should define post-conditions for at least two operations. 
During the development of the project, if you foresee that the size of the VDM++ model 
will be less than 5 pages (or 7.5 pages in case of 
groups of 3 students) or more than 10 
pages (or 15 pages in case of groups of 3 students), you should contact your teacher to 
possibly adjust the scope of the system or the modeling approach being followed. 
\section{Model validation (i.e., testing)}
VDM++ test classes, containing adequate and thoroug
h test cases defined by means of 
operations or traces.  
o
Evidences of test results (passed/failed) and test 
coverage. It is sufficient to present the 
system classes mentioned in 4 painted with coverage
 information. Ideally, 100% cov-
erage should be achieved. 
Optionally, figures of examples exercised in the test cases. 
Requirements  traceability  relationship  between  test  cases  and requirements.  Ideally, 100\%  requirements  coverage  should  be  achieved.  It  is  sufficient  to  indicate  in  com-
ments the requirements that are exercised by each test. 
\section{Model verification (i.e., consistency analysis)}
An example of domain verification, i.e., a proof sketch that a pre-condition of an op-
erator, function or operation is not violated. You 
should present the proof obligation 
generated by the tool and your proof sketch.  
An example of invariant verification, i.e., a proof sketch that the body of an operation 
preserves  invariants.  You  should  present  the  proof obligation  generated  by  the  tool 
and your proof sketch. 
\section{Code generation}
You should try to generate Java code from the VDM++ model and try to execute or test 
the generated code. Here you should describe the steps followed and results achieved. 
\section{Conclusions}
Results achieved  
Things that could be improved 
Division of effort and contributions between team members 
\section{References}
\section{Charge}
\input{specification/Charge.vdmpp.tex}
\section{Client}
\begin{vdmpp}[breaklines=true]
class Client
 
instance variables
 public name : seq1 of char;
 public payment_card : int; 
 
operations
 
(*@
\label{Client:9}
@*)
 public Client : seq1 of char * int ==> Client
  Client(nome, pay_card) == (
  name := nome;
  payment_card := pay_card;
  return self;
 );
 
(*@
\label{getName:16}
@*)
 pure public getName: () ==> seq1 of char
 getName() == return name;
 
(*@
\label{getCardNumber:19}
@*)
 pure public getCardNumber: () ==> int
 getCardNumber() == return payment_card;
 
end Client
\end{vdmpp}

\section{Device}
\begin{vdmpp}[breaklines=true]
class Device
types
 public Plate = seq1 of char;
 public ClientCardNumber = seq1 of nat;
values
-- TODO Define values here
instance variables
 private plate:Plate;
 private cardN:ClientCardNumber;
operations
(*@
\label{Client:11}
@*)
(*@
\label{Device:11}
@*)
  public Device : Plate * ClientCardNumber ==> Device
   Device(pt, cn) == (
    plate := pt;
    cardN := cn;
    return self
   );
functions
-- TODO Define functiones here
traces
-- TODO Define Combinatorial Test Traces here
end Device
\end{vdmpp}

\section{Green\_Way}
\begin{vdmpp}[breaklines=true]
class Green_Way
types
-- TODO Define types here
values
-- TODO Define values here
instance variables
 private sproviders: set of Service_Provider := { };
 private clients: set of Client := {};

operations

--remove um cliente    
(*@
\label{removeClient:13}
@*)
  public removeClient(client: Client) == 
  (  
  clients := clients \ {client};
  )
  pre
   client in set clients;
functions
-- TODO Define functiones here
traces
-- TODO Define Combinatorial Test Traces here
end Green_Way
\end{vdmpp}

\section{Passage}
\begin{vdmpp}[breaklines=true]
class Passage

instance variables
 public client: Client;
 public spot: Spot;
 public time: Time;
 public provider: [Service_Provider];
 public cost: real;
 
operations
(*@
\label{Passage:11}
@*)
 public Passage : Client * Spot * Time * [Service_Provider] * real ==> Passage
  Passage(cl, spt, t, sprovider, cst) == (
   client := cl ;
  spot := spt;
  time := t;
  provider := sprovider;
  cost := cst;
  return self;
 )
 pre
 (
  (
   sprovider = nil
   and 
   isofclass(Highway, spt)
  )
  or
   sprovider <> nil
 );
end Passage
\end{vdmpp}

\section{Protocol}
\input{specification/Protocol.vdmpp.tex}
\section{Service}
\begin{vdmpp}[breaklines=true]
class Service 

operations
(*@
\label{getAllSpots:4}
@*)
 public getAllSpots : () ==> set of Spot
 getAllSpots() == 
    is subclass responsibility;
  
(*@
\label{passa:8}
@*)
 public passa : Client * Spot * Time * [Passage] ==> real
 passa(client, spot, time, last_passage) == 
  is subclass responsibility;
  
end Service
\end{vdmpp}

\section{Service\_Provider}
\begin{vdmpp}[breaklines=true]
class Service_Provider
types
 public SPName = seq1 of char;
 
values
-- TODO Define values here
instance variables
 private name: SPName;
 private services: set of Service := {};
 
operations
(*@
\label{Service:Provider:12}
@*)
 public Service_Provider : SPName ==> Service_Provider
 Service_Provider(n) == (
  name := n;
  return self
 );
 
(*@
\label{getName:18}
@*)
 pure public getName: () ==> SPName
 getName() == return name;
 
 
 
functions
-- TODO Define functiones here
traces
-- TODO Define Combinatorial Test Traces here
end Service_Provider
\end{vdmpp}


\end{document}
